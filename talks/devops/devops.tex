\documentclass{beamer}
\usetheme{metropolis}
\usepackage{pgfpages}
\usepackage{listings}

\ifdefined\NOTES
  \setbeameroption{show only notes}
\fi

\title[Devops Series]{Devops in a nutshell}
\author{Kevin Wang}
\date[DevX 2019]{DevX, 2019}
\logo{\includegraphics[height=0.5cm]{assets/devxlogo.png}}

\begin{document}

\maketitle

\section{Introduction}

\begin{frame}{What is devops?}
  \begin{center}
    a set of application development and operations practices to release
    software
  \end{center}
\end{frame}
\note[itemize]{
  \item application development and operations used to be more segregated
  \item faster paced environment necessitates closer interaction between
    development and operations
  \item devops is a set of practices of developing, testing, deploying, and
    monitoring software to release it at a faster rate
}

\begin{frame}{whoami}
  \textbf{\Large Kevin Wang}

  Tech evangelist

  Tech Director Emeritus @ LA Hacks | Advisor @ DevX

  \texttt{\scriptsize github: @xorkevin}
\end{frame}
\note[itemize]{
  \item I am not an expert
  \item I am passionate about developer experience
  \item I want to share my perspectives from my experiences building,
    deploying, and maintaining applications
}

\section{Networking}
\note[itemize]{
  \item understanding networking is critical to devops
  \item internet applications are fundamentally data transfer
  \item much like rest of CS, is built on layers of abstraction
  \item high level protocols built on top of lower level protocols
  \item protocols are exchangeable, higher level protocols only care that the
    data is delivered in the proper manner by a lower level protocol
}

\subsection{Protocols overview}

\begin{frame}{Internet Protocols}
  \includegraphics[width=\textwidth]{internet_gen.png}
\end{frame}
\note[itemize]{
  \item each layer adds increasing amounts of functionality
  \item link layer built on physical layer, bits organized into frames (e.g.
    ethernet)
  \item link layer enables communication between physically connected machines
  \item internet layer built on link layer, packets (e.g. IP)
  \item internet layer allows networked computers to communicate by address and
    routing
  \item transport layer built on internet layer (e.g. TCP)
  \item TCP provides a bidirectional communication channel with guaranteed data
    delivery as long as the connection is open
  \item application layer built on transport layer (e.g. HTTP)
}

\subsection{Application layer protocols}

\begin{frame}[fragile]{HTTP}
  \scriptsize
  \begin{lstlisting}
GET /index.html HTTP/1.1
Host: www.example.com
  \end{lstlisting}
  \begin{lstlisting}
HTTP/1.1 200 OK
Date: Mon, 23 May 2005 22:38:34 GMT
Content-Type: text/html; charset=UTF-8
Content-Length: 88
Connection: close

<html>
<head>
  <title>Hello World</title>
</head>
<body>
  Hello World
</body>
</html>
  \end{lstlisting}
\end{frame}
\note[itemize]{
  \item http is a resource protocol
  \item most common protocol directly used by a web developer
  \item client can manage resources on a server
  \item verbs: GET, POST, PUT, PATCH, DELETE
  \item does not solely carry html data, can be images, json, etc.
}

\begin{frame}[fragile]{DNS}
  It's always DNS. - /r/sysadmin
  \scriptsize
  \begin{lstlisting}
;; ->>HEADER<<- opcode: QUERY, rcode: NOERROR, id: 14545
;; flags: qr rd ra ; QUERY: 1, ANSWER: 4, AUTHORITY: 0, ADDITIONAL: 0
;; QUESTION SECTION:
;; reddit.com.  IN  A

;; ANSWER SECTION:
reddit.com.  16  IN  A  151.101.65.140
reddit.com.  16  IN  A  151.101.129.140
reddit.com.  16  IN  A  151.101.193.140
reddit.com.  16  IN  A  151.101.1.140

;; AUTHORITY SECTION:
;; ADDITIONAL SECTION:

;; Query time: 0 msec
;; SERVER: 172.16.128.1
;; WHEN: Sat Apr  6 22:49:38 2019
;; MSG SIZE  rcvd: 92
  \end{lstlisting}
\end{frame}
\note[itemize]{
  \item domain name service
  \item manages queries and updates for ip addresses behind domain names
  \item hierarchical structure, root servers (.) -> tld servers (com) ->
    nameserver (reddit)
  \item correct domain and ip necessary for HTTPS
  \item whole host of other application layer protocols: SMTP, SSH, etc.
}

\subsection{OS network stack}

\begin{frame}{Ports}
  \begin{itemize}
    \item OS network endpoint
    \item 16 bit unsigned integer (0-65535)
    \item used by TCP and UDP transport protocols
    \item common ports \begin{itemize}
      \small
      \item http: 80
      \item https: 443
      \item dns: 53
      \item ssh: 22
    \end{itemize}
  \end{itemize}
\end{frame}
\note[itemize]{
  \item os network endpoint
  \item client has ephemeral port allocation for response
}

\begin{frame}{Virtual network}
  \includegraphics[width=\textwidth]{vpn_gen.png}
\end{frame}
\note[itemize]{
  \item just as virtual machine, network can be virtualized
  \item VLAN is a virtual link layer network; used to isolate devices on
    same physical network and control access
  \item VPN virtualizes ip layer network; used to extend a private network
    through a public network; often by companies to control access
  \item virtual network does not care how its ip packet is delivered; it looks
    the same regardless
}

\section{Architecture}

\section{Scaling}

\begin{frame}{Architecture}
\end{frame}
\note[itemize]{
  \item paradigm remains largely the same across architectures
  \item load balancing
  \item service mesh
  \item CDN
  \item edge computing
}

\end{document}
